%%% Do not change from BEGIN to END
%%% BEGIN
\documentclass[a4paper,10pt]{article}

\usepackage[utf8]{inputenc}
\usepackage[T1]{fontenc}
\usepackage[danish,english]{babel}
\usepackage{graphicx}
\usepackage[a4paper,margin=2.7cm]{geometry}

\usepackage[math]{kurier}

\usepackage{amsmath}
\usepackage{amssymb}
\usepackage{amsfonts}
\usepackage{enumerate}

\usepackage{tikz}
\usetikzlibrary{arrows,decorations.pathmorphing,backgrounds,positioning,fit,matrix}


\setlength{\parindent}{0pt}
\setlength{\parskip}{2ex} 


\renewcommand{\vec}[1]{\ensuremath {\mathbf #1}}

\newcommand{\courseid}{DM559/DM545}
\newcommand{\coursename}{Linear and Integer Programming}
\newcommand{\term}{Spring 2017}
\newcommand{\dept}{Department of Mathematics and Computer Science}
%%%%%%%%%%%%%%%%%%%%%%%%%%%%%%%%%%%%%%%%%%%%%%%%%%%%%%%%%%%%

\usepackage{fancyhdr}
\usepackage{lastpage}
\usepackage{listings}
\lstset{language=Python,
  basicstyle=\ttfamily\lst@ifdisplaystyle\footnotesize\fi,%\fontfamily{pzc}\selectfont,%
  stringstyle=\ttfamily,
  commentstyle=\ttfamily,
  showstringspaces=false,
  frame=lines, 
  breaklines=true, tabsize=2,
  extendedchars=true,inputencoding=utf8
}
%\lstavoidwhitepre


\pagestyle{fancy} 
\lhead{{\sc \courseid -- \term }} 
\chead{}
\rhead{\idnr}
\cfoot{Page \thepage\ of \pageref{LastPage}}

\fancypagestyle{plain}{
\lhead{\dept\\
University of Southern Denmark, Odense}
\chead{}
\rhead{\today\\
\idnr}
\lfoot{}
\rfoot{}
%\renewcommand{\headrulewidth}{0pt}
}


\title{\begin{flushleft}
\vspace{-4ex}
\courseid~-- \coursename \\[0.2cm]
{\Large Answers to Obligatory Assignment 1.2, \term \\[3ex]
\hrule}
\end{flushleft}
}


\date{}

\newcommand{\idnr}{mgund15 \& mipil15}
\author{Mikkel Pilegaard(mipil15) \& Mathias Holst Gundersen(mgund15)}

\begin{document}
\maketitle


The first code snippet is the model implementation of the DFJ formulation. This is followed up by two code snippets, which are auxiliary methods.

\begin{lstlisting}
def solve_tsp(points, subtours = []):
    points=list(points)
    V = range(len(points))
    E = [(i,j) for i in V for j in V if i<j]
    E = tuplelist(E)

    m = Model("TSP0")
    m.setParam(GRB.param.Presolve, 0)
    m.setParam(GRB.param.Method, 0)
    m.setParam(GRB.param.MIPGap,1e-7)

    ######### BEGIN: Write here your model for Task 1   

    # Decision Variables
    tmp = []
    
    roads = {}
    for (a,b) in E:
        road = m.addVar(lb=0, ub=1, vtype=GRB.BINARY, name="road"+str((a,b)))
        roads[(a,b)] = road
        tmp += [road * distance(points[a], points[b])]

    # Set the objective function
    m.setObjective( quicksum(tmp), GRB.MINIMIZE )
      
    # Constraint to make sure we have exactly 2 roads for each city
    for i in V:
        m.addConstr( quicksum( roads[(a,b)] for (a,b) in delta([i], V) ), GRB.EQUAL, 2, name="c1_"+str(i)+str((a,b)))
    
    # Constraint to remove subtours
    for i in subtours:
        if len(i) < 2:
            continue
        m.addConstr( quicksum( roads[(a,b)] for (a,b) in Edges(i)) <= len(i) - 1 )
    
    ######### END
    
    m.optimize()
    m.write("tsplp.lp")
    
    if m.status == GRB.status.OPTIMAL:
        print('The optimal objective is %g' % m.objVal)
        m.write("tsplp.sol") # write the solution
        return {(i,j) : roads[(i,j)].x for i,j in E}
    else:
        print "Something wrong in solve_tsplp"
        exit(0)

# Creating the subset of cities
sets = list(powerset(range(len(ran_points))))
# The first element of the list is the empty set and the last element is the full set, hence we remove them.
sets = sets[1:(len(sets)-1)]
solve_tsp(ran_points, sets) Don't solve
\end{lstlisting}

The following methods implements the functions $E$ and $\delta$ from the model description respectively.

\begin{lstlisting}
def Edges(vertices):
    """Takes a set of vertices and returns a set of edges from those vertices. 
    Also edges should be ordered by ij if i < j or ji if j < i"""
    # Every vertex is connected to each other, so therefore we can construct the edges like this
    E = [(i,j) for i in vertices for j in vertices if i<j]
    return E
\end{lstlisting}

\begin{lstlisting}
def delta(v, V):
    """The set of edges in the cut (v, V\{v})"""
    E = [(i,j) for i in v for j in V if i<j]
    D = [(j,i) for i in v for j in V if j<i]
    return E+D
\end{lstlisting}



\newpage
\section*{Task 2}

When removing the integrality constraint and the subtour elimination constraint, the solution contains variables which are not all integers. The result can be observed in figure \ref{fig:plot1}. When the resulting route has non-integer values, we do not have a complete tour and this also means that the matrix cannot be TUM as the b column of the constraints are integer (= 2).
% Man kan se i balance begrænsningen at det skal være lig med 2 og dermed b kollonen er 2

And since the subtour elimination constraint has been removed, if just 2 or more points are grouped up, we will expect a subtour. This means that one can expect some roads that are semi-selected with a fractional value.

\begin{figure}[htb]
\centering
\includegraphics[scale=0.7]{task31.png}
\caption{$\text{TSPLP}_0$}
\label{fig:plot1}
\end{figure}


\newpage
\section*{Task 3}

We solved the relaxed version of the TSP problem and got the result seen in figure \ref{fig:plot1}. We visually inspect the solution and choose the subtour $\texttt{[3,10,11]}$, add this to the solvers parameters,
 and solve the model again. We get the result seen in figure \ref{fig:plot2}; notice the slight change in the left part of the tour.

\begin{figure}[htb]
\centering
\includegraphics[scale=0.7]{task32.png}
\caption{$\text{TSPLP}_1$}
\label{fig:plot2}
\end{figure}

We inspect the result again, find the subtour $\texttt{[3,6,10,11]}$ and add it to our list of subtours. We pass these to the solver and solve again. We repeat this until no more subtours can be found.

The tour can be seen in figure \ref{fig:plot3}, with the final subtour list being:

\begin{lstlisting}
tsplp1 = solve_tsp0(ran_points, 
	[
		(3,10,11),
		(3,6,10,11),
		(3,6,7,10,11),
		(0,4,18),
		(0,4,12,18),
		(3,6,7,10,11,14)
	])
plot_situation(ran_points, tsplp1)
\end{lstlisting}

\begin{figure}[htb]
\centering
\includegraphics[scale=0.7]{task33.png}
\caption{$\text{TSPLP}_6$}
\label{fig:plot3}
\end{figure}

We see that the final tour is in fact a tour, as there is no subtours.

\newpage
\section*{Task 4}

When looking at the separation problem model given, we see that it is not a linear combination (and therefore not solvable by linear programming), due to the variables $z_i$ and $z_j$ being multiplied together. We rewrite this to a linear combination, by observing that both variables are binary, and when multiplied together, they act as the logical AND condition. The AND condition can be rewritten as a linear combination. This is done for our SEP problem using $y$ as the intermediate:

Parameters:
$$ k \in S $$
$$ x^*_{ij} \in \mathbb{R} $$

We are also given the modified graph $G' = \{V', E'\}$.

Variables:
$$ z \in \mathbb{B}^n $$
$$ y_{ij} \in \mathbb{R} \quad \forall ij \in E' $$

Objective function:
$$ \max \sum_{e=ij\in E':i < j} x^*_e y_e - \sum_{i \in V' \setminus \{k\}} z_i $$

Constraints:
$$ z_k = 1 $$
$$ y_{ij} \geq z_i + z_j - 1 \quad \forall ij \in E' $$
$$ y_{ij} \leq z_i,z_j \quad \forall ij \in E' $$
$$ y_{ij} \geq 0 \quad \forall ij \in E' $$



\newpage
\section*{Task 5}

The implementation of SEP:

\begin{lstlisting}
def solve_separation(points, x_star, k):
    points=list(points)
    V = range(len(points))
    Vprime = range(1,len(points))
    E = [(i,j) for i in V for j in V if i<j]
    Eprime = [(i,j) for i in Vprime for j in Vprime if i<j]
    E = tuplelist(E)
    Eprime = tuplelist(Eprime)

    m = Model("SEP")
    m.setParam(GRB.param.OutputFlag,0)
    
    ######### BEGIN: Write here your model for Task 4

    # Decision variables
    z_values = {}
    right_objective = []
    for i in Vprime:
        z_values[i] = m.addVar(lb = 0, ub = 1, vtype = GRB.INTEGER, name = "z_" + str(i))
        if i != k:
            right_objective += [z_values[i]]
    
    y_values = {}
    left_objective = []
    for (i, j) in Eprime:
        if i < j:
            y_values[(i, j)] = m.addVar(lb = 0, ub = 1, vtype = GRB.INTEGER, name = "y_" + str((i, j)))
            left_objective += [x_star[(i, j)] * y_values[(i, j)]]
    
    m.setObjective(quicksum(left_objective) - quicksum(right_objective), GRB.MAXIMIZE)
    
    # Add constraints
    
    m.addConstr(z_values[k] == 1)
    
    for (i, j) in Eprime:
        m.addConstr(y_values[(i, j)] >= z_values[i] + z_values[j] - 1)
        m.addConstr(y_values[(i, j)] <= z_values[i])
        m.addConstr(y_values[(i, j)] <= z_values[j])
            
    
    ######### END
    m.optimize()
    #m.write("sep.lp")
    
    if m.status == GRB.status.OPTIMAL:
        print('Separation problem solved for k=%d, solution value %g' % (k,m.objVal))
        #m.write("sep.sol") # write the solution    
        subtour = filter(lambda i: z_values[i].x>=0.99, z_values)
        return m.objVal, subtour
    else:
        print "Something wrong in solve_tsplp"
        exit(0)
\end{lstlisting}

When running the $\texttt{solve\_separation}$ with a k value between 1 and the amount of cities, then if the resulting objective value is above 0, that means that city k is in a subtour. We observe this fact when solving SEP on our random points.


\newpage
\section*{Task 6}

If we run our random seed with continuous variables, we find a solution after three iterations as seen in \ref{fig:plot4}. 

\begin{lstlisting}
def cutting_plane_alg(points):
    Vprime = range(1,len(points))
    subtours = []
    found = True
    while found:
        lpsol = solve_tsp0(points,subtours)
        # plot_situation(points, lpsol)
        found = False
        tmp_subtours = []
        best_val = float('-inf')
        for k in Vprime:
            value, subtour = solve_separation(points,lpsol,k)
            best_val = value if value > best_val else best_val
            ######### BEGIN: write here the condition. Include a tollerance
            if value > 0.01: 
            ######### END
                found = True
                tmp_subtours += [subtour]
        subtours += tmp_subtours
        print '*'*60
        print "********** Subtours found: ",tmp_subtours," with best value : ",best_val
        print '*'*60
    plot_situation(points, lpsol)
\end{lstlisting}

The solution seems optimal, as the tour is complete.

\begin{figure}[htb]
\centering
\includegraphics[scale=0.7]{task33.png}
\caption{$TSPLP_3$}
\label{fig:plot4}
\end{figure}

\newpage
\section*{Task 9}
 
As seen in figure \ref{fig:plot4}, our tour length is 2652.2 units.

\end{document}
          



